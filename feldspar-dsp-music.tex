\documentclass{scrartcl}

\usepackage[utf8]{inputenc}

\usepackage{natbib}
\usepackage{hyperref}
\usepackage{csquotes}
\usepackage{listings}
\usepackage{graphicx}
\usepackage[colorinlistoftodos]{todonotes}
% \usepackage{parskip}
% \setlength{\parskip}{10pt}
\usepackage{tikz}
\usetikzlibrary{arrows, decorations.markings}
\usepackage{chngcntr}
\counterwithout{figure}{section}


\begin{document}

\begin{titlepage}

\centering
{\scshape\LARGE Master thesis project proposal}

\vspace{0.5cm}
{\huge\bfseries Feldspar for real-time musical DSP
  }

\vspace{2cm}
{\Large Jakob Larsson \texttt{<jakob@karljakoblarsson.com>}}

\vspace{1.0cm}
{\large Suggested Supervisor at CSE: Patrik Jansson }

% \vspace{1.5cm}
\vspace{1.5cm}
{\large Relevant completed courses:}

{\itshape
Types for Programs and Proofs, DAT350 \\
Advanced Functional programming, TDA342 \\
}

\vfill
{\large \today}

\end{titlepage}

% \section{Introduction}

% \section{Goals and Challenges}

% \section{Approach}

Feldspar~\cite{feldspar} is a embedded Haskell DSL designed to develop correct
and portable signal processing algorithms.  Faust~\cite{faust} is a stand-alone
functional DSL for developing sound synthesis and audio processing with focus
on music.  When implementing a DSL there are two different approaches, embedded
or stand-alone. Both with different perks and disadvantages.

There are several advantages to an embedded DSL compared to a stand-alone.
First there is no need to redevelop parsing and syntax checking and second, the
entire host language is available to the user.

Both Feldspar and Faust aim to produce high-performance signal processing code
for use in embedded systems. This is a duplication of effort. We wish to
investigate if it is possible to implement i library similar to the Faust API
on top of Feldspar. This would allow us to reuse all the work that has gone
into Feldspar.

Faust has a more narrow focus, it aims to implement a smaller subset of
algorithms the Feldspar, and therefore should produce more optimized code. Can
we retain these optimization while utilize the ones in Feldspar?

Embedding a DSL in Haskell allows us to use the entire Haskell ecosystem for
our application. For example to use QuickCheck~\cite{claessen_quickcheck_2000}
to verify correctness.

\bibliographystyle{plain}

\bibliography{sfeeca}

\end{document}
