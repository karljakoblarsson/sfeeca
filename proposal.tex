\documentclass{scrartcl}

\usepackage[utf8]{inputenc}

\usepackage{natbib}
\usepackage{hyperref}
\usepackage{csquotes}
\usepackage{listings}
\usepackage{graphicx}
\usepackage[colorinlistoftodos]{todonotes}
% \usepackage{parskip}
% \setlength{\parskip}{10pt}
\usepackage{tikz}
\usetikzlibrary{arrows, decorations.markings}
\usepackage{chngcntr}
\counterwithout{figure}{section}


\begin{document}

\begin{titlepage}

\centering
{\scshape\LARGE Master thesis project proposal\\}
\vspace{0.5cm}
{\huge\bfseries Symbolic Functional FEEC\\}
\vspace{2cm}
{\Large Jakob Larsson <jakob@karljakoblarsson.com>\\}
\vspace{1.0cm}
{\large Suggested Supervisor at CSE: Patrik Jansson \\}
% \vspace{1.5cm}
% {\large Supervisor at Company (if applicable): Name of supervisor, name of company\\}
\vspace{1.5cm}
{\large Relevant completed courses:\par}
{\itshape
Types for Programs and Proofs, DAT350 \\
Advanced Functional programming, TDA342 \\
}
% \vspace{1.5cm}
% {\large Relevant completed courses student 2:\par}
% {\itshape List (course code, name of course)\\}
% \vfill

\vfill
{\large \today\\}
\end{titlepage}


% Draft 3 I guess
%
%

\section{Introduction}

% TODO START HERE Do a complete read through of the pdf-file. Do it now!

% #######
% Briefly describe and motivate the project, and convince the reader of the
% importance of the proposed thesis work.  A good introduction will answer these
% questions: Why is addressing these challenges significant for gaining new
% knowledge in the studied domain? How and where can this new knowledge be
% applied?

Writing correct software is hard.
Functional programming proposes several techniques to make it easier.
One technique is to embed domain specific objects and terms in the programming
language, a so called Domain Specific Language, or DSL.
\cite{van2000domain} Defines a DSL as:

\begin{displayquote}
  A domain-specific language (DSL) is a programming language or executable
  specification language that offers, through appropriate notations and
  abstractions, expressive power focused on, and usually restricted to,
  a particular problem domain.
\end{displayquote}

In this thesis we wish to investigate how a DSL
corresponding the mathematics used can improve an existing software project.
% TODO Should this be the same paragraph?
A DSL is defined over a specific problem domain. The problem domain we choose
in this investigation, is the automatic solving of Partial Differential
Equations. (PDEs)

% Start with FEEC then FEniCS? Or Start with PDE and the FEniCS and then FEEC?
FEniCS~\cite{AlnaesBlechta2015a} is a project which aim to develop a automatic
solver for Partial Differential Equations. The current solver is written in C++
with an interface in Python. A intermediate description of the problems
involved is Finite Element Exterior Calculus, or FEEC.~\cite{arnold2006finite}
There exists a functional implementation of FEEC written in Haskell called
FEECa~\footnote{https://github.com/Airini/FEECa}.  However the implementation
is numerical, we wish to investigate a symbolic implementation.  The thesis is
that a symbolic implementation could both allow for mathematical
simplifications which improves the precision and allow for automated testing of
the program.

% --- Money sentence
% We wish to leverage Haskell's rich type system to implement a symbolic
% version.

% ↓ This is true at least. A FEECa Simplex is bound to `Int`, and the simplex
% is one of the "foundation" types.
% TODO Probably move this to later in the proposal.
The FEECa implementation is bound very hard to the specific type
\lstinline{Int} which in Haskell is a machine-bound integer type. This is
suboptimal.  If the implementation is polymorphic in numeric types, it would be
possibly to use a more specific numeric type which is appropriate to the
specific instance you want to solve.


% TODO Expand on how to model the problems and algebraic structures. What do we
% want to model and what are the benefits and problems with that?
This project aims to implement the abstract concepts of FEEC as a symbolical
DSL for Haskell.  % ↓ Money sentences
We plan to implement a datatype which better captures the semantics of the
underlying mathematical objects.  The thesis is that a more specific datatype
could allow for better solutions.

% TODO Should I mention that taking the magnitude of things is a problem? Since
% square root doesn't fit in "good" algebraic structures. But since it is
% "monotonically increasing/something function", not linear but a one-to-one
% mapping, it's possible to get around by ignoring it. And then just account for
% it at the end.

% We want to model the structure:
% Polynomial → Monoid → Ring → Field → Vector Space
% But how do I describe that (which I don't really understand) in a succinct way?

A symbolic implementation models the structure of the input in the program.
This gives the computer the ability to manipulate not only the values, but the
structure. This gives us the ability to optimize the input expressions to
a structure that is better for the solver. Our thesis is that this will improve
the numerical accuracy and mathematical performance of the solver.


% %%% Why is the planned research a significant step forward? What are the
% %%% scientific challenges. What is hard and non-obvious?
% %%% What problem exists? What are the current solutions and their drawbacks?


\section{Goals and Challenges}
% Describe your contribution with respect to concepts, theory and technical
% goals. Ensure that the scientific and engineering challenges stand out so that
% the reader can easily recognize that you are planning to solve an advanced
% problem.

The goal is the implement a symbolic functional version of Finite Exterior
Calculus.  Then we can optimize the symbolic expressions to improve their
numerical accuracy in the solver. This is because different optimizations are
possible when the computer has knowledge of the expressions structure.


% %%%%%%%%%%%%%%%%%%%%%%%%%%%%%%
% Describe your contribution with respect to concepts, theory and technical
% goals. Ensure that the scientific and engineering challenges stand out so that
% the reader can easily recognize that you are planning to solve an advanced
% problem.


% Challanges
% ==========
There is a functional, but not symbolic, implementation available from
a previous project. This can be a starting point, but is also a liability.
It is impossible to do a complete investigation of the available code at this stage.  This is a uncertainty in the project.


\section{Approach}

% Various scientific approaches are appropriate for different challenges and
% project goals. Outline and justify the ones that you have selected. For
% example, when your project considers systematic data collection, you need to
% explain how you will analyze the data, in order to address your challenges and
% project goals.

% TODO Expand this section.
% But it is hard to have a plan in this stage.
The first step is to find a suitable embedding, in practice this means
designing a hierarchy of data-types with a good correspondent to our problem
domain.  We need to evaluate which algebraic structures gives us the needed
semantics. The next step is to implement this datatype, possibly by extending
the available FEECa implementation.

The evaluation will be done by comparing the results of our Symbolic FEEC
implementation with the FEniCS implementation. If our thesis is correct the
numerical accuracy can be improved without loosing to much performance.
The ambition is to use automated testing, for example with
QuickCheck,~\cite{claessen_quickcheck_2000} to show the we improve accuracy in
many cases.

% TODO End the conclusion with what we wish to achieve, is that a good way?
% Nope.

% Maybe to broad.
The ambition is to show that a good symbolic representation of the mathematical
problem yields significant improvements. Therefore showing that a symbolical DSL
is a powerful tool in numerical software.

\bibliographystyle{plain}

\bibliography{sfeeca}

\end{document}
