\documentclass{scrartcl}

\usepackage[utf8]{inputenc}

\usepackage{natbib}
\usepackage{hyperref}
\usepackage{graphicx}
\usepackage[colorinlistoftodos]{todonotes}
% \usepackage{parskip}
% \setlength{\parskip}{10pt}
\usepackage{tikz}
\usetikzlibrary{arrows, decorations.markings}
\usepackage{chngcntr}
\counterwithout{figure}{section}


\begin{document}

\begin{titlepage}

\centering
{\scshape\LARGE Master thesis project proposal\\}
\vspace{0.5cm}
{\huge\bfseries Symbolic Functional FEEC\\}
\vspace{2cm}
{\Large Jakob Larsson <jakob@karljakoblarsson.com>\\}
\vspace{1.0cm}
{\large Suggested Supervisor at CSE: Patrik Jansson \\}
% \vspace{1.5cm}
% {\large Supervisor at Company (if applicable): Name of supervisor, name of company\\}
\vspace{1.5cm}
{\large Relevant completed courses:\par}
{\itshape
Types for Programs and Proofs, DAT350 \\
Advanced Functional programming, TDA342 \\
}
% \vspace{1.5cm}
% {\large Relevant completed courses student 2:\par}
% {\itshape List (course code, name of course)\\}
% \vfill

\vfill
{\large \today\\}
\end{titlepage}


% Draft 3 I guess
%
%

\section{Introduction}

% Briefly describe and motivate the project, and convince the reader of the
% importance of the proposed thesis work.  A good introduction will answer these
% questions: Why is addressing these challenges significant for gaining new
% knowledge in the studied domain? How and where can this new knowledge be
% applied?

Writing correct software is hard.
Functional programming proposes several techniques to make it easier.
One technique is to embed domain specific objects and terms in the programming
language, a so called Domain Specific Language, or DSL.
In this thesis we wish to investigate how a DSL
corresponding the mathematics used can improve an existing software project.

FEniCS~\cite{AlnaesBlechta2015a} is a project which aim to develop a automatic
solver for Partial Differential Equations. The current solver is written in C++
with an interface in Python. A intermediate description of the problems
involved is Finite Element Exterior Calculus, or FEEC.~\cite{arnold2006finite}
There exists a functional implementation of FEEC written in Haskell called
FEECa~\footnote{https://github.com/Airini/FEECa}.  However the implementation
is numerical, we wish to investigate a symbolic implementation.  The thesis is
that a symbolic implementation could both allow for mathematical
simplifications which improves the precision and allow for automated testing of
the program.


% During that project they have developed Finite Element Exterior
% [> havent mentioned the solver <]
% Calculus FEEC \ref{winter} as a intermediate building block for the solver.  In
% addition to the main implementation in C++ there exists a work-in-progress
% functional implementation in Haskell \ref{feeca}. The current implementation is
% [> numeric vs numerical vs approx <]
% numeric. We wish to leverage Haskell's rich type system to implement a symbolic
% version.


% [> DSL <]
% A symbolic Domain Specific Language would allow many algebraic manipulations
% which could improve the output. It could also enable use of automatic theorem
% provers to guarantee correctness. This could find bugs which would benefit the
% main implementation.

% The current implementation Finite Element Exterior Calculus in Haskell (FEECa)
% uses machine-bound Int64 as the main datatype. Our plan is to implement
% a datatype which better captures the semantics of the underlying mathematical
% objects. The thesis is that a more specific datatype could allow for better
% solutions.

% TODO To say that the compiler could catch purley mathematical errors in the
% input is a very strong statement and maybe not true.
A DSL which leverages Haskell's type system has many advantages in theory.
% What advantages?

The naive approach is to implement the algebraic structures used as types.
polynomials


The theory of Finite Element Exterior Calculus (FEEC) is a mathematical
framework which provides for the discretization of partial differential
equations (PDE).

This project aims to implement the abstract concepts of FEEC as a symbolical
DSL for Haskell. For computations on differential forms.

% Text from README.md
% FEECa is a library implementing the mathematical framework that the theory of
% finite element exterior calculus (FEEC), developed by Arnold, Falk and
% Winther, provides for the discretization of partial differential equations
% (PDEs) that allows for a universal treatment of a large number of physical
% problems.

% FEECa implements the abstract, mathematical concepts of FEEC and provides
% a full-fledged basis form generated and framework for computations on
% differential forms. It handles polynomial differential forms in arbitrary
% dimensions and implements monomial as well as Bernstein bases for
% polynomials.  The package provides functionality to compute bases of the P_r
% L^k and P-_r L^k spaces of finite element spaces based on the geometric
% decomposition proposed by Arnold, Falk and Winther.


% TODO Symbolic vs. Numerical.
% When implementing mathematical algorithms on computer there are two main
% approaches. Model the actuall equations for the computer, this gives the
% possiblity to do advanced things, like automatic simplification or
% diffferentation. But it comes with drawback, much slover computation, memory
% use osv. And it's really hard to do in most programming languages. Or
% inconvenient to do in most mainstream languages.

% Numeric on the otherhand is supported nativley by the computer hardware and is
% much more efficient. However there is many problems, resolution and
% rounding-errors and loss of "debuggability".

% Yada, the person which reads this proposal should already know all this? I can
% just say that symbolic have perks which we want in this implementation.


% TODO Expand on how to model the problems and algebraic structures. What do we
% want to model and what are the benefits and problems with that?

% TODO Should I mention that taking the magnitud of things is a problem? Since
% square root doesn't fit in "good" algebraic structures. But since it is
% "monotonically increasing/somethinf function", not linear but a one-to-one
% mapping, it's possible to get around by ignoring it. And then just account for
% it at the end.

% We want to model the structure:
% Polynomial → Monoid → Ring → Field → Vector Space
% But how do I describe that (which I don't really understand) in a succint way?

% Something about Bernstein polynomials, Barycentric monomials, Simplexes.
% A Bernstein polynomial over a Simplex.



% <!-- unclear, will depend on availible time -->
% It would also be interesting if a more specific data type allow use
% to prove the implementation correct, but that could be for another project.

% The involves thakeing the norm of vectors, this involes taking the square root
% which is hard to fit in a good algebra.

% <!-- Why is the planned research a significant step forward? What are the -->
% <!-- scientific challenges. What is hard and non-obvious? -->
% <!-- What problem exisits? What are the current solutions and their drawbacks? -->



% \section{Context}

% Use one or two relevant and high quality references for providing evidence from
% the literature that the proposed study indeed includes scientific and
% engineering challenges, or is related to existing ones. Convince the reader
% that the problem addressed in this thesis has not been solved prior to this
% project.



\section{Goals and Challenges}
% Describe your contribution with respect to concepts, theory and technical
% goals. Ensure that the scientific and engineering challenges stand out so that
% the reader can easily recognize that you are planning to solve an advanced
% problem.

% Goals
% =====

% To implement a symbolic datatype for the FEECa project. And show that it
% improves results. Or allows for automated testing of correctness proofs.

% TODO This first sentence is the money sentence. Is this really the problem
% we're solving. Is it the most succint problem-formulation?
% The goal is the implement a symbolic functional version of Finite Exterior
% Calculus. Do I need to explain more about what symbolic and functional means
% and the theoretical benefits?
% Do I need to describe FEEC more here? Or is it enough to cite the paper?

The goal is the implement a symbolic functional version of Finite Exterior
Calculus.  Then we want to show that this version improves on the existing
implementation.  Either by showing that the symbolic implementation allows for
algebraic simplifications which improves the results. Or that the Haskell
implementation allows us to use automated verification to prove the
implementation correct.

% TODO Decide if algebraic simplification or automated testing is the main
% focus, discuss with Patrik. AAAAARGH!

% Describe your contribution with respect to concepts, theory and technical
% goals. Ensure that the scientific and engineering challenges stand out so that
% the reader can easily recognize that you are planning to solve an advanced
% problem.


% Challanges
% ==========
% TODO Maybe describe the availible code and info more? More clearly at least.
% There is a half finished but abandond project availible, it is probably in
% worse state the expected.
Since this is a continuation of a previous project, a big challenge is to judge the quality
of the current FEECa implementation.
It is impossible to do a complete investigation of the available code at this stage.
This is a big uncertainty in the project.

% <!-- vague paragraph -->
% <!-- careful with strong statements about bugs -->
% TODO Is this important to mention? Is bugs and code quality really a focus of
% this project? Maybe not, this doesn't give any new or interesting
% information.
It is hard to scientifically evaluate measures to improve the quality of code.
Bugs are hard to use in controlled experiments, therefore it is hard to show
that a new implementation is better than another.



\section{Approach}

% Various scientific approaches are appropriate for different challenges and
% project goals. Outline and justify the ones that you have selected. For
% example, when your project considers systematic data collection, you need to
% explain how you will analyze the data, in order to address your challenges and
% project goals.

% One scientific approach is to use formal models and rigorous mathematical
% argumentation to address aspects like correctness and efficiency. If this is
% relevant, describe the related algorithmic subjects, and how you plan to
% address the studied problem. For example, if your plan is to study the problem
% from a computability aspect, address the relevant issues, such as algorithm and
% data structure design, complexity analysis, etc.  If you plan to develop and
% evaluate a prototype, briefly describe your plans to design, implement, and
% evaluate your prototype by reviewing at most two relevant issues, such as key
% functionalities and their evaluation criteria.

% The design and implementation should specify prototype properties, such as
% functionalities and performance goals, e.g., scalability, memory, energy.
% Motivate key design selection, with respect to state of the art and existing
% platforms, libraries, etc.

% When discussing evaluation criteria, describe the testing environment, e.g.,
% test-bed experiments, simulation, and user studies, which you plan to use when
% assessing your prototype. Specify key tools, and preliminary test-case
% scenarios. Explain how and why you plan to use the evaluation criteria in order
% to demonstrate the functionalities and design goals. Explain how you plan to
% compare your prototype to the state of the art using the proposed test-case
% evaluation scenarios and benchmarks.


% TODO Is usability a goal? Or only "power" of the embedding?
% How do you talk about "usability" in PL research settings?
The first step is to find a suitable embedding or data-types. We need to evaluate
which algebraic structures gives us the best compromise between usability and
the needed semantics. The next step is to develop the FEECa implementation to
use our new data-types.
% TODO Should I really start from the current FEECa or should I start seperatly
% and then integrate that code with my DSL and embeddings? I don't think
% I should included that in the proposal.

The evaluation will be done by comparing the results of our Symbolic FEECa
implementation with the main implementation.
% <!-- more details -->
The ambition is to use automated
testing, for example with QuickCheck,~\cite{claessen_quickcheck_2000}.
% TODO Again, decide if simplification or testing is the main goal of all this.

% TODO this is a total aside, maybe interesting thoughts, but it does not
% belong in the conclusion. This as well, probably not good to include here in
% the proposal.
It would also be very interesting to try to prove the correctness of the
implementation. To do it completely is almost impossible but only proving some
part could potentially find many bugs in the main implementation.


% TODO What is missing from this draft, except for the TODOs above?
% I'm not sure.
% The TODOs are better now, so maybe I only need to fix them now, and then be
% done.


% Ethics
% ======
% <!-- redo this totally -->
% <!-- or maybe skip for now and just take the revision -->
% <!-- should not be part of real proposal, only for CSS -->
% This proposal is close to basic research in programming languages and category
% theory. The most immediate consequences possible is to improve the quality of
% a PDE solver. This can not directly have any negative ethical consequences.
% PDE's occur in the solutions of most interesting physical problems. The
% problems to solve could of course be immoral, but improving the solutions
% a little bit is not going to affect the end result.



\bibliographystyle{plain}

\bibliography{sfeeca}

% Useful citations list
% \cite{arnold2006finite} % FEEC
% \cite{AlnaesBlechta2015a} % The FENICS project 1.5

\end{document}
